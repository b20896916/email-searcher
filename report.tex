\documentclass{article}
\usepackage{CJKutf8}
\usepackage[utf8]{inputenc}
\usepackage{geometry}
\usepackage{fancyhdr}
\usepackage{array}

% Note that you can only edit the content but not the page format
% Please modified below brackets with your team's information.
\def \teamno {10}
\def \teammembers {B09902005, B09902013, B09902096}
\def \bestranking {34 (on 20210625)}
\def \bestscore {72432.12 (on 20210625)}

\geometry{
    a4paper,
    total={7in, 9.5in},
    top=1.1in
}

\renewcommand{\footrulewidth}{0.4pt}
\fontfamily{charter}\selectfont
\pagenumbering{gobble}
\linespread{1.2}

\pagestyle{fancy}
\fancyhf{}
\rhead{Team No.\teamno}
\lhead{DSA Final Project Report}
\lfoot{Ranking: \bestranking \quad Score: \bestscore}
\rfoot{\teammembers}

% If you want to use Chinese, then you need to wrap the Chinese content with \begin{CJK}{UTF8}{bsmi} and \end{CJK}
% For example: \begin{CJK}{UTF8}{bsmi} 中文內容 \end{CJK}

\begin{document}

\section*{Data Structures \& Algorithms}
\textbf{Preprocessing:}\\
Explain what data structures and algorithms you used and why you choose to use them.\\ \\
\textbf{Expression Match:}

We deal with expression first. We first convert the expression (char[]) into a postfix linked list (stack). (How we do this is like what had been taught in Hsuen-Tien Lin's class about stack.)
This takes us $O(l_e)$ time and space to convert, where $l_e$ is the length of expression.

After completing the postfix linked list, we can now use it to see which emails return true with this expression.\\
For every mails, every tokens in the linked list can be viewed as true or false, which means whether the mail content contains this mail.
(Since we use binary search to search a hash value in an email, so to decide true or false for a token, we'll need $O(\log l_l)$ time, where $l_l$ is the length of a letter. So totally, we'll spend $O(n l_e \log l_l)$ time on it, where $n$ is the numer of letters.)
(Note that $l_e = 2048$, $l_l = 100000$.)

Finally, we can calculate the result, and this also takes $O(l_e)$ time.\\ \\
\textbf{Find Similar:}\\
Explain what data structures and algorithms you used and why you choose to use them.\\ \\
\textbf{Group Analysis:}\\
Explain what data structures and algorithms you used and why you choose to use them.\\

\section*{Cost Estimations of Queries}
\textbf{Preprocessing:}\\
Provide and compare the cost estimation of your implementation on each types of queries. Using a table to show difference is encouraged.\\ \\
\textbf{Expression Match:}\\
Our implementation solves Expression Match in $O(n l_e \log l_l)$, where $l_e$ is the length of expression, (about 2048), and $l_l$ is the length of an email (about 100000).\\ \\
\textbf{Find Similar:}\\
Provide and compare the cost estimation of your implementation on each types of queries. Using a table to show difference is encouraged.\\ \\
\textbf{Group Analysis:}\\
Provide and compare the cost estimation of your implementation on each types of queries. Using a table to show difference is encouraged.\\

\begin{center}
 \begin{tabular}{||c c c||} 
 \hline
 \textbf{quiries} & \textbf{time complexity} & \textbf{space complexity} \\ [0.5ex] 
 \hline\hline
 Expression Match & $O(n l_e \log l_l)$ & $O(l_e)$ \\ 
 \hline
 Find Similar & O() & O() \\
 \hline
 Group Analysis & O() & O() \\
 \hline
\end{tabular}
\end{center}

\section*{Scheduling Strategy}
Describe how you design the strategy to answer queries. Does your strategy work as well as you imagined? If (not) so, why?\\ \\


\section*{Additional Notes (*optinal)}
Supply any other efforts/findings you have tried/discovered that (may) improves your implementation.

\end{document}
